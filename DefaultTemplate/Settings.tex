\usepackage[utf8]{inputenc} %basic and default package 

%optional packages:
\usepackage{adjustbox} %can center content wider than the \textwidth
\usepackage{amsmath,amsthm,amsfonts} %for new formulas and symbols
\usepackage{array} % different options for table cell orientation
\usepackage[ngerman,english]{babel} % english is the same as american or USenglish
\usepackage{chngcntr} %for changing the resetting of counters
    \counterwithout{figure}{chapter}
\usepackage{fancyvrb} %for literally replaying text or code 
\usepackage{geometry} %for customizing page layout
    \newgeometry{tmargin=2.5cm,bmargin=2.5cm,lmargin=3cm,rmargin=3cm}
\makeglossary
\usepackage[acronym,xindy,toc]{glossaries}
\usepackage{csquotes}
\usepackage{graphicx} %for including figures and images
\usepackage{glossaries-extra}

\usepackage{lipsum} %for generic text generation
\usepackage{makecell} % allows nice manual configuration of cells with linebreaks in \thead and \makecell with alignments
\usepackage{multirow} %makes it possible to have bigger cells over multiple rows in a table
\usepackage{pdfpages} % for including multiple pages of pdfs
\usepackage{setspace} %for adjusting spacing
    \onehalfspacing
\usepackage{siunitx} % for physical accurate units and other numerical presentations
\usepackage{subfig} % for having figures side by side
\usepackage{tablefootnote} % for footnotes in tables as \tablefootnote
\usepackage{pdfpages}
\usepackage{etoolbox}%for \includefilesfrom command
\usepackage{threeparttable} % another way to add footnotes as \tablenotes with \item [x] <your footnote> after setting \tnote{x}
\usepackage{tikz} %for mathematical drawings
\usepackage{titlesec}
    \titleformat{\chapter}{\normalfont\huge\bfseries}{\thechapter.}{20pt}{\huge\bfseries}
\usepackage{xcolor} %for writing in different colors
\usepackage[scaled]{helvet}
\renewcommand{\familydefault}{\sfdefault}
%packages needed for bibliography
\usepackage[style=apa,backend=biber]{biblatex}
\addbibresource{References.bib}
\newcommand{\includefilesfrom}[1]{%
    \foreach \index in {1,2,3,4,5,6,7,8,9,10,11,12,13,14,15,16,A,B,C,D,E,F,G,H,I,J,K,L,M,N,O,P,Q,R,S,T,U,V,W,X,Y,Z}{%
    \IfFileExists{#1/\index}{%
    \input{#1/\index}}%
    }{%
    }%
}%
\usepackage{float}%for \begin{figure}[H] foreced position
\usepackage{caption} % needed to customize captions
\usepackage{chngcntr} % needed to customize counter behavior

\makeatletter
\let\c@lofdepth\relax
\let\c@lotdepth\relax
\makeatletter

\usepackage{tocloft} % needed to customize the list of figures
\captionsetup[figure]{
    labelformat=simple, 
    labelsep=space,
    textfont=normalfont,
    labelfont={bf}, % makes the "Fig." part bold
    name=Fig.
}
% Redefine the figure numbering format to "Fig. 1.1"
\renewcommand{\thefigure}{\thechapter.\arabic{figure}}
\counterwithin{figure}{chapter}
% Customize the List of Figures appearance
\renewcommand{\cftfigfont}{\textbf{ Fig.~}} % Makes "Fig." bold in the List of Figures
\renewcommand{\cftfigpresnum}{\bfseries} % Makes the figure number bold in the List of Figures
\renewcommand{\cftfigaftersnum}{\normalfont} % Adds a space after the figure number

% Redefine the table caption label to "Tab." and make it bold
\captionsetup[table]{
    labelformat=simple, 
    labelsep=space,
    textfont=normalfont,
    labelfont={bf}, % makes the "Tab." part bold
    name=Tab.
}
% Redefine the table numbering format to "Tab. 1.1"
\renewcommand{\thetable}{\thechapter.\arabic{table}}
% Ensure tables are numbered within chapters
\counterwithin{table}{chapter}
% Customize the List of Tables appearance
\renewcommand{\cfttabfont}{\textbf{ Tab.~}} % Makes "Tab." bold in the List of Tables
\renewcommand{\cfttabpresnum}{\bfseries} % Makes the table number bold in the List of Tables
\renewcommand{\cfttabaftersnum}{\normalfont} % Switch back to normal font for caption text


% Define a custom command for creating glossary entries with bold terms
\newcommand{\newboldglossary}[3]{%
  \newglossaryentry{#1}{%
    name={\textbf{#2}},  
    description={#3},
    text={\textbf{#2}}  
  }%
}

